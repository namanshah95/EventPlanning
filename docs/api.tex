\documentclass{article}

\usepackage{amsfonts}
\usepackage{amsmath}
\usepackage{amssymb}
\usepackage{color}
\usepackage[margin=0.5in]{geometry}

\def\arraystretch{1.5}
\renewcommand{\labelitemi}{$\diamond$}

\begin{document}

Notes:
\begin{enumerate}
    \item Output fields marked with * can be passed into the HTTP request header to filter results.
    \item Parameters in red are required.
    \item The following HTTP methods signify the following:
    \begin{enumerate}
        \item GET is used to retrieve data from the database.
        \item POST is used to insert new data into the database.
        \item PUT is used to update existing data in the database.
        \item DELETE is used to remove existing data from the database.
    \end{enumerate}
\end{enumerate}

\begin{itemize}
\item \textbf{GET /entities/} \smallskip \\
\begin{tabular}{|p{4cm}|p{12.85cm}|} \hline
Description & Gets an entity from the database. \\ \hline
HTTP Header Params & \begin{tabular}{|p{4cm}|p{8cm}|}
    \texttt{Email} & The email address of this entity as stored in Firebase. \\ \hline
    \texttt{Name} & The name of this entity as stored in Firebase. \\
    \end{tabular} \\ \hline
HTTP Body Params & None. \\ \hline
Output & \begin{tabular}{|p{4cm}|p{8cm}|}
    \texttt{FIREBASE OUTPUT} & Refer to the fields stored in Firebase for details on what this route returns. \\ \hline
    \texttt{entity} & PostgreSQL record PK for this entity. \\ \hline
    \texttt{ext\_firebase\_id} & Firebase record key for this entity. \\
    \end{tabular} \\ \hline
\end{tabular} \bigskip
\item \textbf{GET /entities/\{entity\}} \smallskip \\
\begin{tabular}{|p{4cm}|p{12.85cm}|} \hline
Description & Gets an entity from the Firebase database. This finds the external Firebase ID associated with the specified entity PK, and then uses the external Firebase ID to get all of the entity's credentials from Firebase. \\ \hline
HTTP Header Params & None. \\ \hline
HTTP Body Params & None. \\ \hline
Output & \begin{tabular}{|p{4cm}|p{8cm}|}
    \texttt{FIREBASE OUTPUT} & Refer to the fields stored in Firebase for details on what this route returns. \\ \hline
    \texttt{ext\_firebase\_id} & Firebase record key for this entity. \\
    \end{tabular} \\ \hline
\end{tabular} \bigskip
\item \textbf{GET /entity/\{entity\}/events/} \smallskip \\
\begin{tabular}{|p{4cm}|p{12.85cm}|} \hline
Description & Gets all events that a specified entity is either the owner of or a guest of, and the roles they have for those events. Multiple records will be returned for the same event if the entity has more than one role for that event. \\ \hline
HTTP Header Params & None. \\ \hline
HTTP Body Params & None. \\ \hline
Output & \begin{tabular}{|p{4cm}|p{8cm}|}
    \texttt{*entity} & The PK of the guest. \\ \hline
    \texttt{*estimated\_budget} & The budget that the guest should dedicate to this role. \\ \hline
    \texttt{*event} & The PK of the event. \\ \hline
    \texttt{*event\_entity\_role} & Record PK. \\ \hline
    \texttt{*role} & The PK of the role. \\ \hline
    \texttt{*role\_name} & The name of the role. \\ \hline
    \texttt{event\_name} & The name of the event. \\
    \end{tabular} \\ \hline
\end{tabular} \bigskip
\item \textbf{GET /event/\{event\}/guests/} \smallskip \\
\begin{tabular}{|p{4cm}|p{12.85cm}|} \hline
Description & Returns all guests attending a specific event and their roles. The result set does NOT include the owner of the event. Multiple records will be returned for the same guest if they have more than one role. \\ \hline
HTTP Header Params & None. \\ \hline
HTTP Body Params & None. \\ \hline
Output & \begin{tabular}{|p{4cm}|p{8cm}|}
    \texttt{*entity} & The PK of the guest. \\ \hline
    \texttt{*estimated\_budget} & The budget that the guest should dedicate to this role. \\ \hline
    \texttt{*event} & The PK of the event. \\ \hline
    \texttt{*event\_entity\_role} & Record PK. \\ \hline
    \texttt{*role} & The PK of the role. \\ \hline
    \texttt{*role\_name} & The name of the role. \\ \hline
    \texttt{event\_name} & The name of the event. \\
    \end{tabular} \\ \hline
\end{tabular} \bigskip
\item \textbf{GET /event/\{event\}/guests/\{guest\}} \smallskip \\
\begin{tabular}{|p{4cm}|p{12.85cm}|} \hline
Description & Returns a specific guest attending a specific event and their roles. No results will be returned if the specified guest is the owner. Multiple records will be returned for the same guest if they have more than one role. This is identical to calling \texttt{GET /event/\{event\}/guests/?entity={guest}}. \\ \hline
HTTP Header Params & None. \\ \hline
HTTP Body Params & None. \\ \hline
Output & \begin{tabular}{|p{4cm}|p{8cm}|}
    \texttt{*entity} & The PK of the guest. \\ \hline
    \texttt{*event} & The PK of the event. \\ \hline
    \texttt{*event\_entity\_role} & Record PK. \\ \hline
    \texttt{*role} & The PK of the role. \\ \hline
    \texttt{*role\_name} & The name of the role. \\ \hline
    \texttt{event\_name} & The name of the event. \\
    \end{tabular} \\ \hline
\end{tabular} \bigskip
\item \textbf{GET /event/\{event\}/owner/} \smallskip \\
\begin{tabular}{|p{4cm}|p{12.85cm}|} \hline
Description & Gets the owner of a specific event and their roles. Multiple records will be returned for the owner if they have more than one role. \\ \hline
HTTP Header Params & None. \\ \hline
HTTP Body Params & None. \\ \hline
Output & \begin{tabular}{|p{4cm}|p{8cm}|}
    \texttt{*entity} & The PK of the guest. \\ \hline
    \texttt{*estimated\_budget} & The budget that the guest should dedicate to this role. \\ \hline
    \texttt{*event} & The PK of the event. \\ \hline
    \texttt{*event\_entity\_role} & Record PK. \\ \hline
    \texttt{*role} & The PK of the role. \\ \hline
    \texttt{*role\_name} & The name of the role. \\ \hline
    \texttt{event\_name} & The name of the event. \\
    \end{tabular} \\ \hline
\end{tabular} \bigskip
\item \textbf{GET /event/\{event\}/roles/} \smallskip \\
\begin{tabular}{|p{4cm}|p{12.85cm}|} \hline
Description & Returns all roles needed for a specific event. \\ \hline
HTTP Header Params & None. \\ \hline
HTTP Body Params & None. \\ \hline
Output & \begin{tabular}{|p{4cm}|p{8cm}|}
    \texttt{*description} & The event-level description, or default description if an event-level description doesn't exist, of the role needed by the event. \\ \hline
    \texttt{*estimated\_budget} & The amount of money that should be dedicated to this role. \\ \hline
    \texttt{*event\_needed\_role} & Record PK. \\ \hline
    \texttt{*needed\_role} & The PK of the role needed by the event. \\ \hline
    \texttt{*needed\_role\_name} & The name of the role needed by the event. \\ \hline
    \texttt{*quantity\_needed} & The number of people needed to perform this role. \\ \hline
    \texttt{event} & The PK of the event. \\
    \end{tabular} \\ \hline
\end{tabular} \bigskip
\item \textbf{GET /event/\{event\}/roles/\{role\}} \smallskip \\
\begin{tabular}{|p{4cm}|p{12.85cm}|} \hline
Description & Gets a specific needed role for a specific event. \\ \hline
HTTP Header Params & None. \\ \hline
HTTP Body Params & None. \\ \hline
Output & \begin{tabular}{|p{4cm}|p{8cm}|}
    \texttt{*description} & The event-level description, or default description if an event-level description doesn't exist, of the role needed by the event. \\ \hline
    \texttt{*estimated\_budget} & The amount of money that should be dedicated to this role. \\ \hline
    \texttt{*quantity\_needed} & The number of people needed to perform this role. \\ \hline
    \texttt{event} & The PK of the event. \\ \hline
    \texttt{event\_needed\_role} & Record PK. \\ \hline
    \texttt{needed\_role} & The PK of the role needed by the event. \\ \hline
    \texttt{needed\_role\_name} & The name of the role needed by the event. \\
    \end{tabular} \\ \hline
\end{tabular} \bigskip
\item \textbf{GET /events/} \smallskip \\
\begin{tabular}{|p{4cm}|p{12.85cm}|} \hline
Description & Gets all events. \\ \hline
HTTP Header Params & None. \\ \hline
HTTP Body Params & None. \\ \hline
Output & \begin{tabular}{|p{4cm}|p{8cm}|}
    \texttt{*created} & The exact moment this event was created. \\ \hline
    \texttt{*end\_time} & The date and time the event ends at. \\ \hline
    \texttt{*event} & Record PK. \\ \hline
    \texttt{*name} & The name of the event. \\ \hline
    \texttt{*start\_time} & The date and time the event begins at. \\
    \end{tabular} \\ \hline
\end{tabular} \bigskip
\item \textbf{GET /events/\{event\}} \smallskip \\
\begin{tabular}{|p{4cm}|p{12.85cm}|} \hline
Description & Gets a specific event. \\ \hline
HTTP Header Params & None. \\ \hline
HTTP Body Params & None. \\ \hline
Output & \begin{tabular}{|p{4cm}|p{8cm}|}
    \texttt{created} & The exact moment this event was created. \\ \hline
    \texttt{end\_time} & The date and time the event ends at. \\ \hline
    \texttt{event} & Record PK. \\ \hline
    \texttt{name} & The name of the event. \\ \hline
    \texttt{start\_time} & The date and time the event begins at. \\
    \end{tabular} \\ \hline
\end{tabular} \bigskip
\item \textbf{GET /roles/} \smallskip \\
\begin{tabular}{|p{4cm}|p{12.85cm}|} \hline
Description & Gets all roles from the database. \\ \hline
HTTP Header Params & None. \\ \hline
HTTP Body Params & None. \\ \hline
Output & \begin{tabular}{|p{4cm}|p{8cm}|}
    \texttt{*description} & The default description of the role. \\ \hline
    \texttt{*name} & The name of the role. \\ \hline
    \texttt{*role} & Record PK. \\
    \end{tabular} \\ \hline
\end{tabular} \bigskip
\item \textbf{GET /roles/\{role\}} \smallskip \\
\begin{tabular}{|p{4cm}|p{12.85cm}|} \hline
Description & Gets a specific role from the database. This is equivalent to calling \texttt{GET /roles/?role=\{role\}}. \\ \hline
HTTP Header Params & None. \\ \hline
HTTP Body Params & None. \\ \hline
Output & \begin{tabular}{|p{4cm}|p{8cm}|}
    \texttt{*description} & The default description of the role. \\ \hline
    \texttt{*name} & The name of the role. \\ \hline
    \texttt{*role} & Record PK. \\
    \end{tabular} \\ \hline
\end{tabular} \bigskip
\item \textbf{POST /entities/} \smallskip \\
\begin{tabular}{|p{4cm}|p{12.85cm}|} \hline
Description & Adds a new entity. \\ \hline
HTTP Header Params & None. \\ \hline
HTTP Body Params & \begin{tabular}{|p{4cm}|p{8cm}|}
    \texttt{\textcolor{red}{ext\_firebase\_id}} & The external Firebase ID corresponding to the newly-created entity. \\
    \end{tabular} \\ \hline
Output & \begin{tabular}{|p{4cm}|p{8cm}|}
    \texttt{entity} & The PK of the entity that was just created. \\
    \end{tabular} \\ \hline
\end{tabular} \bigskip
\item \textbf{POST /event/\{event\}/entities/\{entity\}/roles/} \smallskip \\
\begin{tabular}{|p{4cm}|p{12.85cm}|} \hline
Description & Adds a role to a specific entity attending a specific event. Note that the specified entity can be either the owner or a guest. Additionally, because this creates a new \texttt{event\_entity\_role} record, if an entity is not already associated with an event, calling this route will associate them with the event. \\ \hline
HTTP Header Params & None. \\ \hline
HTTP Body Params & \begin{tabular}{|p{4cm}|p{8cm}|}
    \texttt{estimated\_budget} & The budget that the entity should allocate to performing the role. \\ \hline
    \texttt{\textcolor{red}{role}} & The role that should be given to the specified entity attending the specified event. \\
    \end{tabular} \\ \hline
Output & \begin{tabular}{|p{4cm}|p{8cm}|}
    \texttt{event\_entity\_role} & The PK of the record that was created. \\
    \end{tabular} \\ \hline
\end{tabular} \bigskip
\item \textbf{POST /event/\{event\}/guests/} \smallskip \\
\begin{tabular}{|p{4cm}|p{12.85cm}|} \hline
Description & Adds a guest to a specific event. This creates a new \texttt{event\_entity\_role} record with the specific event, specified entity, and the Guest role. \\ \hline
HTTP Header Params & None. \\ \hline
HTTP Body Params & \begin{tabular}{|p{4cm}|p{8cm}|}
    \texttt{\textcolor{red}{entity}} & The PK of the entity who should be added to the specified event as a guest. \\
    \end{tabular} \\ \hline
Output & \begin{tabular}{|p{4cm}|p{8cm}|}
    \texttt{event\_entity\_role} & The PK of the record that was created. \\
    \end{tabular} \\ \hline
\end{tabular} \bigskip
\item \textbf{POST /event/\{event\}/roles/} \smallskip \\
\begin{tabular}{|p{4cm}|p{12.85cm}|} \hline
Description & Adds a needed role to a specific event. \\ \hline
HTTP Header Params & None. \\ \hline
HTTP Body Params & \begin{tabular}{|p{4cm}|p{8cm}|}
    \texttt{description} & The event-level description of the role needed by this event. \\ \hline
    \texttt{estimated\_budget} & The amount of money that should be dedicated to this role. \\ \hline
    \texttt{\textcolor{red}{needed\_role}} & The PK of the role needed by the event. \\ \hline
    \texttt{quantity\_needed} & The number of people needed to perform this role. \\
    \end{tabular} \\ \hline
Output & \begin{tabular}{|p{4cm}|p{8cm}|}
    \texttt{event\_needed\_role} & The PK of the record that was created. \\
    \end{tabular} \\ \hline
\end{tabular} \bigskip
\item \textbf{POST /events/} \smallskip \\
\begin{tabular}{|p{4cm}|p{12.85cm}|} \hline
Description & Creates an event. \\ \hline
HTTP Header Params & None. \\ \hline
HTTP Body Params & \begin{tabular}{|p{4cm}|p{8cm}|}
    \texttt{\textcolor{red}{end\_time}} & The date and time the event ends at. \\ \hline
    \texttt{\textcolor{red}{name}} & The name of the event. \\ \hline
    \texttt{\textcolor{red}{owner}} & The entity PK of the owner of the event being created. \\ \hline
    \texttt{\textcolor{red}{start\_time}} & The date and time the event begins at. \\
    \end{tabular} \\ \hline
Output & \begin{tabular}{|p{4cm}|p{8cm}|}
    \texttt{event} & The PK of the event that was created. \\
    \end{tabular} \\ \hline
\end{tabular} \bigskip
\item \textbf{POST /roles/} \smallskip \\
\begin{tabular}{|p{4cm}|p{12.85cm}|} \hline
Description & Adds a new role to the database. This does not add a role to any specific event, nor does it add a role to any specific entity. \\ \hline
HTTP Header Params & None. \\ \hline
HTTP Body Params & \begin{tabular}{|p{4cm}|p{8cm}|}
    \texttt{description} & The default description of the role to create. \\ \hline
    \texttt{\textcolor{red}{name}} & The name of the role to create. \\
    \end{tabular} \\ \hline
Output & \begin{tabular}{|p{4cm}|p{8cm}|}
    \texttt{role} & The PK of the role that was created. \\
    \end{tabular} \\ \hline
\end{tabular} \bigskip
\item \textbf{PUT /event/\{event\}/entities/\{entity\}/roles/\{role\}} \smallskip \\
\begin{tabular}{|p{4cm}|p{12.85cm}|} \hline
Description & Updates the role information for a specific entity at a specific event. Note that the entity may be either the owner or a guest of the event. \\ \hline
HTTP Header Params & None. \\ \hline
HTTP Body Params & \begin{tabular}{|p{4cm}|p{8cm}|}
    \texttt{estimated\_budget} & The new amount of money that the entity should dedicate to this role. \\
    \end{tabular} \\ \hline
Output & \begin{tabular}{|p{4cm}|p{8cm}|}
    \texttt{event\_entity\_role} & The PK of the record that was updated. \\
    \end{tabular} \\ \hline
\end{tabular} \bigskip
\item \textbf{PUT /event/\{event\}/roles/\{role\}} \smallskip \\
\begin{tabular}{|p{4cm}|p{12.85cm}|} \hline
Description & Updates a specific needed role for a specific event. \\ \hline
HTTP Header Params & None. \\ \hline
HTTP Body Params & \begin{tabular}{|p{4cm}|p{8cm}|}
    \texttt{description} & The new event-level description of the role needed by this event. \\ \hline
    \texttt{estimated\_budget} & The new amount of money that should be dedicated to this role. \\ \hline
    \texttt{quantity\_needed} & The new number of people needed to perform this role. \\
    \end{tabular} \\ \hline
Output & \begin{tabular}{|p{4cm}|p{8cm}|}
    \texttt{event\_needed\_role} & The PK of the recorded that was updated. \\
    \end{tabular} \\ \hline
\end{tabular} \bigskip
\item \textbf{PUT /events/\{event\}} \smallskip \\
\begin{tabular}{|p{4cm}|p{12.85cm}|} \hline
Description & Updates an event. \\ \hline
HTTP Header Params & None. \\ \hline
HTTP Body Params & \begin{tabular}{|p{4cm}|p{8cm}|}
    \texttt{end\_time} & The date and time the event ends at. \\ \hline
    \texttt{name} & The name of the event. \\ \hline
    \texttt{start\_time} & The date and time the event begins at. \\
    \end{tabular} \\ \hline
Output & \begin{tabular}{|p{4cm}|p{8cm}|}
    \texttt{event} & The PK of the event that was updated. \\
    \end{tabular} \\ \hline
\end{tabular} \bigskip
\item \textbf{PUT /roles/\{role\}} \smallskip \\
\begin{tabular}{|p{4cm}|p{12.85cm}|} \hline
Description & Updates an existing role in the database. \\ \hline
HTTP Header Params & None. \\ \hline
HTTP Body Params & \begin{tabular}{|p{4cm}|p{8cm}|}
    \texttt{description} & The new default description the role should have. \\ \hline
    \texttt{name} & The new name the role should have. \\
    \end{tabular} \\ \hline
Output & \begin{tabular}{|p{4cm}|p{8cm}|}
    \texttt{role} & The PK of the role that was updated. \\
    \end{tabular} \\ \hline
\end{tabular} \bigskip
\item \textbf{DELETE /event/\{event\}/guests/\{guest\}} \smallskip \\
\begin{tabular}{|p{4cm}|p{12.85cm}|} \hline
Description & Completely removes a specified guest from an event. This removes every \texttt{event\_entity\_role} record that has the specified guest as the entity. \\ \hline
HTTP Header Params & None. \\ \hline
HTTP Body Params & None. \\ \hline
Output & \begin{tabular}{|p{4cm}|p{8cm}|}
    \texttt{event\_entity\_role} & The PK of the record that was deleted. \\
    \end{tabular} \\ \hline
\end{tabular} \bigskip
\item \textbf{DELETE /event/\{event\}/guests/\{guest\}/roles/\{role\}} \smallskip \\
\begin{tabular}{|p{4cm}|p{12.85cm}|} \hline
Description & Removes a role from an event guest. \\ \hline
HTTP Header Params & None. \\ \hline
HTTP Body Params & None. \\ \hline
Output & \begin{tabular}{|p{4cm}|p{8cm}|}
    \texttt{event\_entity\_role} & The PK of the guest role that was deleted. \\
    \end{tabular} \\ \hline
\end{tabular} \bigskip
\item \textbf{DELETE /event/\{event\}/roles/\{role\}} \smallskip \\
\begin{tabular}{|p{4cm}|p{12.85cm}|} \hline
Description & Removes a needed role from an event. \\ \hline
HTTP Header Params & None. \\ \hline
HTTP Body Params & None. \\ \hline
Output & \begin{tabular}{|p{4cm}|p{8cm}|}
    \texttt{event\_needed\_role} & The PK of the record that was deleted. \\
    \end{tabular} \\ \hline
\end{tabular} \bigskip
\item \textbf{DELETE /events/\{event\}} \smallskip \\
\begin{tabular}{|p{4cm}|p{12.85cm}|} \hline
Description & Deletes an event. \\ \hline
HTTP Header Params & None. \\ \hline
HTTP Body Params & None. \\ \hline
Output & \begin{tabular}{|p{4cm}|p{8cm}|}
    \texttt{event} & The PK of the event that was deleted. \\
    \end{tabular} \\ \hline
\end{tabular} \bigskip
\item \textbf{DELETE /roles/\{role\}} \smallskip \\
\begin{tabular}{|p{4cm}|p{12.85cm}|} \hline
Description & Deletes an existing role from the database. \\ \hline
HTTP Header Params & None. \\ \hline
HTTP Body Params & None. \\ \hline
Output & \begin{tabular}{|p{4cm}|p{8cm}|}
    \texttt{role} & The PK of the role that was deleted. \\
    \end{tabular} \\ \hline
\end{tabular} \bigskip
\end{itemize}

\end{document}
